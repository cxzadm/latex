%%%%%%%%%%%%%%%%%%%%%%%%%%%%% Define Article %%%%%%%%%%%%%%%%%%%%%%%%%%%%%%%%%%
\documentclass{article}
%%%%%%%%%%%%%%%%%%%%%%%%%%%%%%%%%%%%%%%%%%%%%%%%%%%%%%%%%%%%%%%%%%%%%%%%%%%%%%%

%%%%%%%%%%%%%%%%%%%%%%%%%%%%% Using Packages %%%%%%%%%%%%%%%%%%%%%%%%%%%%%%%%%%
\usepackage{geometry}
\usepackage{graphicx}
\usepackage{amssymb}
\usepackage{amsmath}
\usepackage{amsthm}
\usepackage{empheq}
\usepackage{mdframed}
\usepackage{booktabs}
\usepackage{lipsum}
\usepackage{color}
\usepackage{psfrag}
\usepackage{pgfplots}
\usepackage{bm}
%%%%%%%%%%%%%%%%%%%%%%%%%%%%%%%%%%%%%%%%%%%%%%%%%%%%%%%%%%%%%%%%%%%%%%%%%%%%%%%

% Other Settings

%%%%%%%%%%%%%%%%%%%%%%%%%% Page Setting %%%%%%%%%%%%%%%%%%%%%%%%%%%%%%%%%%%%%%%
\geometry{a4paper}

%%%%%%%%%%%%%%%%%%%%%%%%%% Define some useful colors %%%%%%%%%%%%%%%%%%%%%%%%%%
\definecolor{ocre}{RGB}{243,102,25}
\definecolor{mygray}{RGB}{243,243,244}
\definecolor{deepGreen}{RGB}{26,111,0}
\definecolor{shallowGreen}{RGB}{235,255,255}
\definecolor{deepBlue}{RGB}{61,124,222}
\definecolor{shallowBlue}{RGB}{235,249,255}
%%%%%%%%%%%%%%%%%%%%%%%%%%%%%%%%%%%%%%%%%%%%%%%%%%%%%%%%%%%%%%%%%%%%%%%%%%%%%%%

%%%%%%%%%%%%%%%%%%%%%%%%%% Define an orangebox command %%%%%%%%%%%%%%%%%%%%%%%%
\newcommand\orangebox[1]{\fcolorbox{ocre}{mygray}{\hspace{1em}#1\hspace{1em}}}
%%%%%%%%%%%%%%%%%%%%%%%%%%%%%%%%%%%%%%%%%%%%%%%%%%%%%%%%%%%%%%%%%%%%%%%%%%%%%%%

%%%%%%%%%%%%%%%%%%%%%%%%%%%% English Environments %%%%%%%%%%%%%%%%%%%%%%%%%%%%%
\newtheoremstyle{mytheoremstyle}{3pt}{3pt}{\normalfont}{0cm}{\rmfamily\bfseries}{}{1em}{{\color{black}\thmname{#1}~\thmnumber{#2}}\thmnote{\,--\,#3}}
\newtheoremstyle{myproblemstyle}{3pt}{3pt}{\normalfont}{0cm}{\rmfamily\bfseries}{}{1em}{{\color{black}\thmname{#1}~\thmnumber{#2}}\thmnote{\,--\,#3}}
\theoremstyle{mytheoremstyle}
\newmdtheoremenv[linewidth=1pt,backgroundcolor=shallowGreen,linecolor=deepGreen,leftmargin=0pt,innerleftmargin=20pt,innerrightmargin=20pt,]{theorem}{Theorem}[section]
\theoremstyle{mytheoremstyle}
\newmdtheoremenv[linewidth=1pt,backgroundcolor=shallowBlue,linecolor=deepBlue,leftmargin=0pt,innerleftmargin=20pt,innerrightmargin=20pt,]{definition}{Definition}[section]
\theoremstyle{myproblemstyle}
\newmdtheoremenv[linecolor=black,leftmargin=0pt,innerleftmargin=10pt,innerrightmargin=10pt,]{problem}{Problem}[section]
%%%%%%%%%%%%%%%%%%%%%%%%%%%%%%%%%%%%%%%%%%%%%%%%%%%%%%%%%%%%%%%%%%%%%%%%%%%%%%%

%%%%%%%%%%%%%%%%%%%%%%%%%%%%%%% Plotting Settings %%%%%%%%%%%%%%%%%%%%%%%%%%%%%
\usepgfplotslibrary{colorbrewer}
\pgfplotsset{width=8cm,compat=1.9}
%%%%%%%%%%%%%%%%%%%%%%%%%%%%%%%%%%%%%%%%%%%%%%%%%%%%%%%%%%%%%%%%%%%%%%%%%%%%%%%

%%%%%%%%%%%%%%%%%%%%%%%%%%%%%%% Title & Author %%%%%%%%%%%%%%%%%%%%%%%%%%%%%%%%
\title{\color{red}Estructura de un Documento en \LaTeX}

\author{Bryam Fernando Cabrera Sarmiento}

\date{20 de Julio de 2023}
%%%%%%%%%%%%%%%%%%%%%%%%%%%%%%%%%%%%%%%%%%%%%%%%%%%%%%%%%%%%%%%%%%%%%%%%%%%%%%%
% ... (paquetes y configuraciones)

\title{\color{red}Estructura de un Documento en \LaTeX}
\author{Bryam Fernando Cabrera Sarmiento}
\date{20 de Julio de 2023}

\begin{document}
    \maketitle
    
    \begin{abstract}
        La clase utilizada en esta actividad es \verb|article| y el tamaño de letra es de 11 puntos. Debe definirse un documento \verb|pdf| que sea idéntico a este documento.
        
        Para la elaboración de listas, puede que necesites consultar la referencia proporcionada en la sección \textbf{entornos}.
    \end{abstract}
    
    \section{Introducción}\label{sec:1}
    En la sección~\ref{SB1}, se realiza una práctica de Listas.
    
    \subsection{Listas}\label{SB1}
    \begin{itemize}
        \item Primer Elemento
        \begin{itemize}
            \item Primer Subelemento
            \item Segundo Subelemento
        \end{itemize}
        \item Segundo Elemento
        \item Tercer Elemento
        \item \ldots
    \end{itemize}
    
    \subsection{Listas Enumeradas}\label{SB2}
    \begin{enumerate}
        \item Primer elemento
        \begin{enumerate}
            \item Primer Subelemento
            \item Segundo Subelemento
        \end{enumerate}
        \item Segundo Elemento
        \item Tercer Elemento
    \end{enumerate}
    
    \subsection{Descripción de Elementos}\label{SB3}
    \begin{description}
        \item[Rojo] Color que caracteriza el peligro.
        \item[Azul] Color del cielo.
    \end{description}
    
    \section{Matemáticas}\label{Sec:2}
    En la sección~\ref{EC1}, realizaremos ecuaciones matemáticas. La fórmula de la teoría de la relatividad es $E=mc^2$.
    
    \subsection{Exponentes e Índices}\label{EC1}
    \begin{equation}
        k_{n+1}=n^2 + k_n^2 - k_{n-1}
    \end{equation}
    En la ecuación~\ref{EC1}, se observa el uso de Exponentes e índices.
    
    \subsection{Raíces Cuadradas}
    \begin{equation}
        \sqrt{\frac{a}{b}}
    \end{equation}
    \begin{equation}
        \sqrt[n]{1+x+x^2+x^3+\ldots}
    \end{equation}
    \begin{equation}
        \frac{n!}{k!(n-k)!}
    \end{equation}
    
    \subsection{Letras Griegas}
    \begin{equation}
        \alpha \mu_1 + \beta \mu_2 = v
    \end{equation}
    La suma de los $n$ primeros números enteros positivos es $\frac{n(n+1)}{2}$, es decir, $1+2+\dots+n(n+1)$. Utilizando la notación sumatoria, lo anterior se escribiría como $\sum_{i=1}^{n}=\frac{n(n+1)}{2}$; fórmula que volveremos a escribir en modo resaltado.
    \[
    \sum_{i=1}^{n}=\frac{n(n+1)}{2}
    \]
    
    \subsection{Matriz}
    \[
    \begin{matrix}
        a_{11} & a_{12} & a_{13} & \ldots & a_{1n} \\
        a_{21} & a_{22} & a_{23} & \ldots & a_{2n} \\
        \hdotsfor[2]{5} \\
        a_{n1} & a_{n2} & a_{n3} & \ldots & a_{nn}
    \end{matrix}
    \]
    \begin{equation}
        \begin{matrix}
            a & b \\
            c & d
        \end{matrix}
    \end{equation}
    \begin{equation}
        \begin{pmatrix}
            a & b \\
            c & d
        \end{pmatrix}
    \end{equation}
    \begin{equation}
        \begin{bmatrix}
            a & b \\
            c & d
        \end{bmatrix}
    \end{equation}
    \begin{equation}
        \begin{Bmatrix}
            a & b \\
            c & d
        \end{Bmatrix}
    \end{equation}
    \begin{equation}
        \begin{vmatrix}
            a & b \\
            c & d
        \end{vmatrix}
    \end{equation}
    \begin{equation}
        \begin{Vmatrix}
            a & b \\
            c & d
        \end{Vmatrix}
    \end{equation}
    \begin{equation}
        \begin{pmatrix}
            1 & 0 & \cdots & 0 \\
            0 & 1 & \cdots & 0 \\
            \vdots & \vdots & \ddots & \vdots \\
            0 & 0 & \cdots & 1
        \end{pmatrix}
    \end{equation}
    
    \subsection{Casos}
    \[
    \operatorname{sing} x=
    \begin{cases}
        1, & x>0 \\
        0, & x=0 \\
        -1, & x<0
    \end{cases}
    \]
    
    \subsection{Ecuaciones en varias líneas}
    \begin{eqnarray}
        (a+b)^2-(a-b)^2 &=& \\
        (a+b)(a+b)-(a-b)(a-b) &=& \\
        (a^2+2ab+b^2)-(a^2-2ab+b^2) &=& 4ab
    \end{eqnarray}
    \begin{align}
        (a+b)^2-(a-b)^2 &= (a+b)(a+b)-(a-b)(a-b) \\
        &= (a^2+2ab+b^2)-(a^2-2ab+b^2) \\
        &= 4ab
    \end{align}
    
    \section{Figuras}
    \begin{figure}[h]
        \centering
        \includegraphics[width=40mm,height=10mm]{logo}
        \caption{Logo UPS}
        \label{F1}
    \end{figure}
    En la Fig.~\ref{F1}, se muestra el Logo de la Universidad.
    \begin{figure}[h]
        \centering
        \includegraphics[width=0.5\textwidth, angle=180]{logo.png}
        \caption{Logo UPS}
        \label{F1}
    \end{figure}
    
    \section{Tablas}
    \begin{table}[h]
        \centering
        \begin{tabular}{ccc}
            Nombre & Apellido & Nota \\ \hline \hline
            Erwin  & Sacoto   & 34   \\
            Erwin1 & Sacoto 1 & 36   \\
            Jairo  & Sacoto   & 35   \\ \hline \hline
        \end{tabular}
        \caption{Notas}
        \label{T1}
    \end{table}
    En la tabla~\ref{T1}, se muestran las notas de los estudiantes.
    \begin{table}[h]
        \centering
        \begin{tabular}{c|c|c}
            \multicolumn{3}{c}{Cuadro de Notas} \\ \hline
            Nombre & Apellido & Nota \\
            Erwin  & Sacoto   & 34   \\
            Jairo  & Sacoto   & 35   \\ \hline
        \end{tabular}
    \end{table}
    
    \section{Cómo citar}\label{sec:cita}
    En la sección~\ref{sec:cita}, se indica cómo citar en un documento~\cite{uno}.
    Los autores en~\cite{uno} indican que \ldots
    En~\cite{uno}, se define \ldots
    La Ingeniería de Software es \ldots, tal como se indica en~\cite{dos}.
    En los diferentes estudios~\cite{uno,dos}.
    
    \bibliographystyle{ieeetr}
    \bibliography{bib}
    
\end{document}
